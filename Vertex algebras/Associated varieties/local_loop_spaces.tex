\section{Local formal loop spaces}
    \subsection{Jet spaces of schemes}
        \begin{definition}[Jet spaces] \label{def: jet_spaces}
            For any prestack $Y$, let us write $J^{\infty}(Y)$ to mean the prestack whose functor of points is given by:
                $$J^{\infty}(Y)(\Spec R) := \Maps_{\Pre\Stk}(\Spf R[\![z]\!], Y)$$
            $J^{\infty}(Y)$ is usually called the \textbf{jet space} or space of \textbf{infinite jets} of $Y$.
            
            Some authors also use the term \say{arc space} to refer to $J^{\infty}(Y)$ and reserve the term \say{jet space} to refer to the truncations $J^n(Y)$ as in remark \ref{remark: representability_of_jet_spaces}, but we find this somewhat confusing. 

            The notation $Y[\![z]\!]$ is also used sometimes in place of $J^{\infty}(Y)$, especially when $Y$ is a group scheme or when so-called \say{factorisation structures}\footnote{It is still premature to discuss what these are. We would only like to clarify the various notations that can be commonly found throughout the literature.} are involved.
        \end{definition}
        \begin{remark}[Representability of jet spaces] \label{remark: representability_of_jet_spaces}
            Immediately from the fact that:
                $$\Spf R[\![z]\!] := \indlim_{n \geq 1} \Spec R[z]/z^n$$
            one sees that:
                $$J^{\infty}(Y)(\Spec R) \cong \projlim_{n \geq 1} \Maps_{\Pre\Stk}(\Spec R[z]/z^n, Y)$$
            For each $n \geq 1$, write $J^n(Y)$ to mean the prestack whose functor of points is given by:
                $$J^n(Y)(\Spec R) := \Maps_{\Pre\Stk}(\Spec R[z]/z^n, Y)$$
            One readily sees that if $Y$ is an algebraic stack (or even better, an algebraic space or even a scheme), then $J^n(Y)$ will also be an algebraic stack (respectively, algebraic space and scheme). Furthermore, notice that the transition maps:
                $$J^{n + 1}(Y) \to J^n(Y)$$
            are representable by affine morphisms of schemes in such cases, $J^{\infty}(Y)$ will also be an algebraic stack (respectively, algebraic space and scheme) should $Y$ itself be so to begin with. Also, the schemes $J^n(Y)$ are typically referred to either as the schemes of \textbf{$n$-jets}, of \textbf{$n^{th}$ order jets}, or of truncated \textbf{jets of order $n$}; to our knowledge, all of these terms are equally common.
    
            Also, because the transition maps $J^{n + 1}(Y) \to J^n(Y)$ are affine, so is the canonical map $\pi: J^{\infty}(Y) \to Y$. In particular, this means that for every open immersion $U \hookrightarrow Y$, we shall have that:
                $$U \x_{Y, \pi} J^{\infty}(Y) \cong J^{\infty}(U)$$
            More generally, for any \textit{subcanonical}\footnote{So in practice, fpqc or coarser.} topology $\tau$ on $\Sch$ defining a small site $Y_{\tau}$ (with terminal object $Y$), we can instead consider $\tau$-covering maps $U \to Y$. This is an easy consequence of the fact that limits commute. 
        \end{remark}
        
        If $R$ is a local $\bbC$-algebra with maximal ideal $\m$, then $R[\![z]\!]$ will also be a local ring, with maximal ideal generated by $\m$ and $z$. Using this fact, we are able to check the following facts about jet spaces of schemes. 
        \begin{lemma}[Basic properties of jet spaces of schemes] \label{lemma: basic_properties_of_jet_spaces}
            Let $Y$ be a scheme.
            \begin{enumerate}
                \item If $S$ is another scheme, then:
                    $$\Maps_{\Sh( \pt_{\Zar} )}( S, J^{\infty}(Y) ) \cong \Maps_{\LRS}( (|S|, \scrO_S[\![z]\!]), Y )$$
                \item As a consequence of the above, one has that:
                    $$J^{\infty}(Y)(\Spec R) \cong \Maps_{\Sch}(\Spec R[\![z]\!], Y)$$
                \item Additionally, if $Y$ is of finite type (respectively, unramified/smooth/\'etale), then each $J^n(Y)$ will also be of finite type (respectively, \textit{formally} unramified/smooth/\'etale). Moreoever, if $Y$ is affine on top of being of finite type, then $J^{\infty}(Y)$ will also be affine.
            \end{enumerate}
        \end{lemma}
        The proofs are routine, so we will not be spelling them out (and we assume that the reader is familiar with Zariski descent). That said, let us note that to prove the third statement (when $Y$ is affine), the reader should recall that $\Sch^{\aff} \cong \Pro(\Sch^{\aff, \ft})$.

        \begin{lemma}[Gluing jet spaces] \label{lemma: gluing_jet_spaces}
            
        \end{lemma}

    \subsection{Nil-Laurent series and local formal loop spaces}
        We begin with the definition, which is a nice analogue of definition \ref{def: jet_spaces}.
        \begin{definition}[Local formal loop spaces] \label{def: local_formal_loop_spaces}
            The \textbf{local formal loop space} of a prestack $Y$ is the prestack $Y(\!(z)\!)$ whose functor of points is given by:
                $$Y(\!(z)\!)(\Spec R) := \Maps_{\Pre\Stk}(\Spec R(\!(z)\!), Y)$$
        \end{definition}
    
        Before we jump into analysing the structure of formal loop spaces, let us point the reader towards a certain subtlety.
        
        Let $\Lambda$ be a (complete) Noetherian local $\bbC$-algebra with residue field $\bbC$\footnote{Eventually, we will be thinking of $\bbC$ as the residue field at a point $x$ of a base Noetherian scheme $X$ (in fact, a smooth projective curve more often than not), and $\Lambda$ as the adic completion of the stalk $\scrO_{X, x}$.}; also, let $\C_{\Lambda, \bbC}$ denote the category of Artinian local $\Lambda$-algebras with residue fields isomorphic to $\bbC$. 

        Next, consider the following:
            $$
                \begin{aligned}
                    Y(\!(z)\!)(\Spec R) & := \Maps_{\Sch}(\Spec R(\!(z)\!), Y)
                    \\
                    & \cong \Maps_{\Sch}(\Spec( \indlim_{n \geq 0} z^{-n} R[\![z]\!] ), Y)
                    \\
                    & \cong \indlim_{n \geq 0} \Maps_{\Sch}( \Spec R[\![z^{1 - n}]\!], Y )
                \end{aligned}
            $$
        for every $R \in \Ob( \C_{\Lambda, \bbC} )$. Now, consider for a moment, $Y$ being a \textit{proper} scheme (so \textit{a priori}, of finite type and qs). Using the Valuative Criterion for Properness (cf. \cite[\href{https://stacks.math.columbia.edu/tag/0BX5}{Tag 0BX5}]{stacks}), one see thus that there is a bijection:
            $$Y|_{\C_{\Lambda, \bbC}}( \Spec R[\![z]\!] ) \xrightarrow[]{\cong} Y|_{\C_{\Lambda, \bbC}}( \Spec R(\!(z)\!) )$$
        for every $R \in \Ob( \C_{\Lambda, \bbC} )$. As such, as sheaves on $(\C_{\Lambda, \bbC}^{\op})_{\Zar}$, the functors $J^{\infty}(Y)$ and $Y(\!(z)\!)$ are actually isomorphic, and hence we see that the functor $Y(\!(z)\!)$ is not able to \say{see} the negative portion $z^{-1} R[z^{-1}] \subset R(\!(z)\!)$ when $Y$ is proper. Since proper schemes are rather common in practice, this is a bit of a problem. That said, notice that $Y(\!(z)\!)$ is nevertheless an ind-scheme \textit{a priori}.

        One way out of this problem is to consider, instead of formal Laurent series, so-called \say{nil-Laurent series}. 
        \begin{definition}[Nil-Laurent series] \label{def: nil_laurent_series}
            Let $R$ be a commutative ring and write $R_{\red} := R/\Nil(R)$ for the quotient of $R$ by its nilradical, consisting of nilpotent elements of $R$. The $R$-algebra of \textbf{nil-Laurent series} with coefficients in $R$ is then the nilpotent extension of $R[\![z]\!]$ given by:
                $$R(\!(z)\!)_{\nil} := R[\![z]\!] \oplus z^{-1} \Nil(R)[z^{-1}]$$
        \end{definition}
        The following properties of nil-Laurent series are easy to prove, but we reckon they are worth recording nevertheless.
        \begin{lemma}
            Let $R$ be a commutative ring.
            \begin{enumerate}
                \item $R(\!(z)\!)_{\nil}$ is an $R$-subalgebra of $R(\!(z)\!)$. 
                \item For any element $f(z) \in R(\!(z)\!)_{\nil}$, the following statements are equivalent:
                \begin{enumerate}
                    \item $f(z) \in R(\!(z)\!)_{\nil}^{\x}$.
                    \item $f(z) \in R_{\red}[\![z]\!]^{\x}$.
                    \item If $f(z) := \sum_{i \gg -\infty} a_i z^i$, then $a_0 \in R^{\x}$.
                \end{enumerate}
                In particular, one sees that $R(\!(z)\!)$ will \textit{not} be a field even when $R$ is a field.

                Moreoever, if $R$ is a local ring with maximal ideal $\m$, then $R(\!(z)\!)_{\nil}$ will also be a local ring, with maximal ideal $\m_{\red}[\![z]\!] + z R_{\red}[\![z]\!]$, where $\m_{\red} := \m/\nil(R)$.
                \item $R(\!(z)\!)_{\nil} \to R[\![z]\!]$ is a cofiltered limit of thickenings, namely:
                    $$R(\!(z)\!)_{\nil} \cong z^{-1} \Nil(R)[z^{-1}] \cong \projlim_{n \geq 1} \Nil(R)[z^{-1}]/z^{-n}$$
            \end{enumerate}
        \end{lemma}